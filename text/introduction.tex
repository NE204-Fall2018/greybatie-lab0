
Gamma rays are quantized electromagnetic radiation produced
by nuclear transitions. These uncharged particles cannot directly ionize or
excite atoms in material it traverses through, making them difficult to be
detected directly. Therefore, it is necessary that some fraction of the incident
photon's energy to be transferred to an electron in the absorbing
material. These fast electrons can then induce excitations or ionization which
 provide valuable information about the nature of the incident gamma rays.

The sensitive volume of a radiation detector serves as a
gamma-ray spectrometer, measuring the intensity and energy of incident gamma rays.
It is advantageous for this region to be composed of solid material, to increase
the probability of a photon interacting inside of it. High purity germanium detectors,
a common type of semi-conductor detector, employ a highly absorptive germanium crystal,
 in addition to being compact, and offering
fast timing characteristics \cite{knoll}.

These devices must be calibrated so that the signal produced
corresponds to the correct incident radiation energy. Most calibration
procedures involve using a known gamma ray source to assign the output
voltage of the detector with the corresponding known gamma ray energy. Once
a detector is properly calibrated it is capable of measuring unknown sources
to better understand a radiation field of interest.
