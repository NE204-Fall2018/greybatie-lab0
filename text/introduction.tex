
Gamma rays are electromagnetic radiation produced
by nuclear transitions. These photons are uncharged and create
no direct ionization or excitation of the material it passes through.
It is therefore necessary for that some or all of the incident
photon's energy to be transferred for an electron in the absorbing
material It is these fast electrons that provide information about
the nature of the incident gamma rays.


The sensitive volume of a radiation detector that serves as a
gamma-ray spectrometer must have a reasonably high probability information
for incident photons to produce one or more fast electrons \cite{knoll}.
In addition, these devices must be calibrated so that the signal produced
corresponds to the correct incident radiation energy. Most calibration
procedures involve using a known gamma ray source to assign the output
voltage of the detector with the corresponding known gamma ray energy. Once
a detector is properly calibrated it is capable of measuring unknown sources
to better understand a radiation field of interest.
